\documentclass[11pt]{article}

\usepackage[utf8]{inputenc}
\usepackage[a4paper, margin=3cm]{geometry}
\usepackage{amssymb}
\usepackage{mathtools}
\usepackage{xfrac}
\usepackage{hyperref}
\usepackage{pgfplots}
\usepackage{color}

\newcommand{\mypath}{.}
\newcommand\worries[1]{\textcolor{red}{#1}}
\newcommand\Mean[1]{\mathbb{E}\!\left[#1\right]}
\newcommand\Var[1]{\mathbb{V}\!\left[#1\right]}
\newcommand\Cov[2]{\mathrm{Cov}\!\left[#1,#2\right]}
\newcommand\Gauss[2]{\mathcal{N}\!\left({#1},\,{#2}\right)}
\newcommand\GP[2]{\mathcal{GP}\!\left({#1},\,{#2}\right)}
\DeclareMathOperator*{\argmin}{arg\,min}

\title{Correlation Based Surface Wave Tomography}
\author{Stefan Mauerberger}

\begin{document}
\maketitle
\begin{abstract}
    As a proof of concept, a synthetic test in surface wave tomography is presented.
    The application in mind is the prediction of group velocities associated with a narrow frequency band.
    Pseudo travel times are generated from the reference model and are corrupted by normal noise.
    Correlations are derived from the squared exponential kernel serving as covariance for the velocity model.
    In a potential future application ambient noise records may serve as data basis.
    The following is just a brief description of the experiment and a summary of formulas I am using.
\end{abstract}

\section{Introduction}
\input{def_example}

This is going to be a synthetic test.
The method of Gaussian process regression is applied together with an underlying correlation structure for surface wave tomography.
For the time being I skip on a description of the modeling method.
Details may be found in my PhD thesis.

The idea in my mind is to use ambient noise records for surface wave tomography.
The average slowness amongst station pairs are obtained from cross correlations.
To avoid the difficulty of dispersion, only a narrow frequency band shall be considered.
Slownesses are picked from bandpass filtered correlograms.
Diffraction is neglected to simplify things even further.
Then, the ray path between stations is approximated by a great circle segment.
Travel times amongst stations are given by distance times the average slowness.

\section{Setting}

The reference model $\tilde v$ is of constant velocity $4\,\sfrac{km}s$ perturbed by two blobs; just large scale anomalies.
At their maximum those blobs are showing a deviation of about $\pm 1\%$ and a characteristic length of $\approx\!50\,km$.
To proceed with a partly realistic setting the station geometry is borrowed from a subset of the ScanArray (DOI: 10.14470/6T569239).
Due to computational limitations I am just considering $\SFWnst$ stations.
Figure~\ref{fig:path_coverage} depicts the reference velocity model and the path coverage.

\begin{figure}
    \centering
    \input{fig_path_coverage.pgf}
    \caption{Atop of the reference velocity model $\tilde v$ the station geometry and path coverage are outlined. }
    \label{fig:path_coverage}
\end{figure}

Pseudo observations are denoted by $d$ and are generated from the reference velocity model considering all pairs of stations.
Integrating the slowness along the great circle segment yields synthetic travel times.
To simulate measurement noise the set of synthetic travel times is corrupted by normal noise of standard deviation $\sigma_\varepsilon = \SFWepsilon$\,.
In total that are $\SFWnobs$ observations.


\section{Line Integral along a Great Circle Segment}

In our oversimplified setting the ray path is given by the shortest distance between two points $a$, $b$ at the Earth's surface.
The central angle is given by
\begin{equation}
    \cos \sphericalangle_a^b %= \frac{a \cdot b}{|a||b|}
    = \sin\phi_a\sin\phi_b + \cos\phi_a\cos\phi_b \cos(\lambda_a - \lambda_b)
\end{equation}
with $\phi$ and $\lambda$ referring to latitude and longitude.
To integrate along the ray path, a parametrizing equation of the grate circle passing through $a$ and $b$ is required.
With respect to a Cartesian coordinate frame such a parametrization is of the form
\begin{equation}
    r(t) = a \cos t + w \sin t
\end{equation}
with $t \in [0:2\pi]$.
The point $w$ is determined by
\begin{align}
    b &= r(t = \sphericalangle_a^b) &
    &\leadsto &
    w &= \frac{b - a \cos \sphericalangle_a^b}{\sin \sphericalangle_a^b}
\end{align}
and the segment between $a$ and $b$ is given by $C= \{r(t) \mid t \in [0, \sphericalangle_a^b]\}$.
%
Now we can calculate travel times by carrying out the line integral along the great circle segment
\begin{equation}
    \mathrm T_{a,b}[v]
    = \int_C \frac1{v(r)} \, \mathrm d r
    = \int_0^{\sphericalangle_a^b} \frac 1{v(r(t))} |\acute r(t)| \, \mathrm d t
\end{equation}
where $v$ refers to a velocity model.
The length of a line element is given by
\begin{equation}
    |\acute c(t)| = |w \cos t - u \sin t| = ??? = r_E
\end{equation}
where $r_E$ refers to the Earth's radius.


\section{Discretization}

As we do not consider diffraction an adoptive discretization is not needed.
To integrate travel times Simpson's rule is used, a fixed samples quadrature scheme.
Therefore, it is necessary to discretize the great circle segment for all station pairs.
I am using a highly irregular discretization which is handy particularly for numeric integration.
All the segments are sampled at almost the same spacing.
The increment is chosen as
\begin{equation}
    \Delta\sphericalangle = \frac 1 \SFWminsamples \min\left\{ \sphericalangle_a^b \mid a,b \in S \right\}
    \approx \SFWdeltaangle \,{}^\circ
\end{equation}
i.e.~half the shortest inter-station distance.
The resulting grid -- totaling to a $\SFWnpts$ points -- is depicted in Figure~\ref{fig:correlation}.


\section{Correlation Kernel}

Typically, the Gaussian kernel is expressed in terms of the euclidean distance.
As we consider waves traveling along the surface, the great circle segment is the desired measure of distance.
At the surface, the shortest distance between two points $x$, $y$ is given by
\begin{equation}
    d(x,y) = \sphericalangle_x^y r_E
\end{equation}
with values in $[0, \pi r_E]$.
The so called great circle distance $d$ shall serve as metric and the correlation kernel of choice reads
\begin{equation}
    K(x,y) = \tau^2 \exp\!\left(-\frac 12 \frac{d(x,y)^2}{\ell^2}\right)
\end{equation}
with characteristic length $\ell$ and variance $\tau^2$.
%That Kernel has an upper and lower bound
%\begin{equation}
%    1 \geq K(x_i,x_j) \geq \tau^2 e^{-\frac12\frac{R^2\pi^2}{\ell^2}} > 0
%\end{equation}
%To proof $K$ is PD consider arbitrary wights $\alpha_i \in \mathbb R$ and locations $x_i \in \mathbb S_{r_E}$ and we have
%\begin{equation}
%    \sum_{ij} \alpha_i K(x_i, x_j) \alpha_j >
%    \varepsilon \sum_{ij} \alpha_i \alpha_j =
%    \varepsilon \sum_i 1 \alpha_i \sum_j 1 \alpha_j =
%    \langle 1,\alpha \rangle^2 > 0
%\end{equation}
%where $\varepsilon$ refers to the lower bound of $K$.

\subsection{Poisson Kernel}
\worries{Matthias suggested to use a Poisson type kernel.
\begin{equation}
    u(x) \propto \iint \frac{R-|x|}{|x-\xi|^3} u(\xi) \, \mathrm d \xi
\end{equation}
Consider white noise at some reference radius $R$ and the covariance is
\begin{equation}
    K(x,y) \propto \iint \frac{R-|x|}{|x-\xi|^3} \frac{R-|y|}{|y-\xi|^3} \, \mathrm d \xi
\end{equation}
the choice of $R$ determines the regularity. }

\begin{figure}
    \centering
    \input{fig_correlation.pgf}
    \caption{Example for correlations amongst travel time and a constant velocity model (parameters are listed in Eq.~\ref{eq:parameters}).
        In addition the overall discretization of is shown. }
    \label{fig:correlation}
\end{figure}


\section{A\,Priori Model}

Let assume an a\,priori model of constant mean velocity.
Then, four parameters need to be determined before any inversion may be carried out.
That are the error level $\varepsilon$, the kernels characteristic length $\tau$, variance $\sigma$ and the a\,priori mean velocity $\mu_V$.
Just to get some result I am using the following values:
\begin{align}\label{eq:parameters}
    \varepsilon &= \SFWepsilon \, s \;,&
    \tau &= \SFWtau \, \sfrac ms \;,&
    \ell &= \SFWell \, m \;,&
    \mu_C &= \SFWmuCpri \, \sfrac ms \;.
\end{align}

Certainly, it were preferable to infer those four parameters from data.
I don't see any difficulties using an ordinary ML estimate.
The log likelihood function reads
\begin{equation}
    -\frac 12 \left( n \ln 2\pi + n \ln |\Var D| + (D - \Mean D)^T \dots  \right)
\end{equation}


\section{Formulas}

Linearizing the travel time, the correlation amongst velocity model and travel time reads
\begin{equation}
    \Cov{V(x)}{T_C[V]} \approx -\int_C \frac{K_V(x,y)}{\mu_V(y)^2}  \mathrm d y
\end{equation}
An example is shown in Figure~\ref{fig:correlation}.
Use a successive approach to build the posterior
\begin{alignat}{3}
    \Mean{V|d_{i+1}} &\approx \Mean{V|d_i} &&+ \Cov{V}{T_C[V]}\Var{D}^{-1} \left(d_{i+1}-\Mean{T_C[V]}\right)
    \\
    \Var{V|d_{i+1}}  &\approx \Var{V|d_i}  &&- \Cov{V}{T_C[V]}\Var{D}^{-1}  \Cov{T_C[V]}{V}
\end{alignat}
The result is shown in Figure~\ref{fig:example}. The parameters used are listed in Equation~\ref{eq:parameters}.
The small scale patches are suggesting to increase the characteristic length.

\begin{figure}
    \centering
    \input{fig_example.pgf}
    \caption{The left panel shows the posterior mean velocity model whereas the left depicts posterior variances. }
    \label{fig:example}
\end{figure}


\section{Next Steps}

\begin{itemize}
    \item It were interesting to see how the misfit successively evolves
    \item Estimate parameters
    \item In that context the average slownesses obtained from cross correlations were the better observable.
        Considering travel times is going round in circles.
        The ray path -- i.e.~the distance -- in general depends on the velocity and is unknown.
        There actually is no need for travel times.
        The average slowness
        \begin{equation}
            \bar p = \left(\int_C |\acute r(t)| \mathrm d t \right)^{-1} \int_C \frac 1 {v(r(r))} |\acute r(t)| \mathrm d t
        \end{equation}
        is a decent observable.

\end{itemize}



\end{document}

