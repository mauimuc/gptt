\documentclass[11pt]{article}

\usepackage[utf8]{inputenc}
\usepackage[a4paper, margin=3cm]{geometry}
\usepackage{amssymb}
\usepackage{mathtools}
\usepackage{xfrac}
\usepackage{hyperref}
\usepackage{pgfplots}
\usepackage{color}

\newcommand{\mypath}{.}
\newcommand\worries[1]{\textcolor{red}{#1}}

\title{Correlation Based Surface Wave Tomography}
\author{Stefan Mauerberger}

\begin{document}
\maketitle
\begin{abstract}
    Just a summary of formulas I use \dots
\end{abstract}

\section{Setting}
\input{def_example}

The idea in my mind is to use ambient noise records for surface wave tomography.
Average slownesses amongst stations are obtained from cross correlations.
Only a narrow frequency band shall be considered to work around dispersion.
Diffraction is neglected to simplify things even further.
Then, the ray path between stations is approximated by a great circle segment.
Travel times amongst stations are given by distances divided by average slowness.

To proceed with a partly realistic setting the station geometry is borrowed from the NORSAR Array.
\worries{Due to memory limitations I just consider a subset of \SFWnobs\ stations.}
The reference model $\tilde C$ is of constant velocity $4\,\sfrac{km}s$ perturbed by two blobs.
At their maximum those blobs are showing a deviation of about $\pm 50\,\sfrac ms$ and a characteristic length of about $50\,km$.
Figure~\ref{fig:path_coverage} depicts the reference velocity model with the path coverage atop.

\begin{figure}
    \centering
    \input{fig_path_coverage.pgf}
    \caption{Path coverage of the NORSA Array}
    \label{fig:path_coverage}
\end{figure}

Synthetic travel times are obtained by integrating the reference slowness along the great circle segment.
\worries{Any pair of combinations of the NORSA Array is used to \dots\
The set of pseudo observations is generated by corrupting synthetic travel times by normal noise.}


\section{Line Integral along a Great Circle Segment}
In the oversimplified setting the ray path is given by the shortest distance between two points $u$, $v$ on the surface of a sphere.
The central angle is given by
\begin{align}
    \cos \sphericalangle_u^v &= \frac{u \cdot v}{|u||v|} &
    &\text{or}&
    \sin \sphericalangle_u^v &= \frac{|u \times v|}{|u||v|}
    \; .
\end{align}
To integrate along the ray path, a parametrizing equation of the grate circle passing through $u$ and $v$ is required.
With respect to a Cartesian coordinate frame such a parametrization is of the form
\begin{equation}
    r(t) = u \cos t + w \sin t
\end{equation}
with $t \in [0:2\pi$] and to be determined $w$.
For $w$ we find
\begin{align}
    v &= r(t = \sphericalangle_u^v) &
    &\leadsto &
    w &= \frac{v - u \cos \sphericalangle_u^v}{\sin \sphericalangle_u^v}
    \; .
\end{align}
The segment between $u$ and $v$ is given by $C= \{r(t) \mid t \in [0, \sphericalangle_u^v]\}$.
Travel times are calculated by carrying out the line integral along the great circle segment
\begin{equation}
    T_{u,v}[s]
    = \int_C \frac1{c(r)} \mathrm d r
    = \int_0^{\sphericalangle_u^v} \frac 1{c(r(t))} |\acute r(t)| \mathrm d t
\end{equation}
where $c$ refers to a velocity model.
The length of a line element on a great circle is given by
\begin{equation}
    |\acute c(t)| = |w \cos t - u \sin t| = ??? = r_E
    \; .
\end{equation}


\section{Discretization}

As we do not consider diffraction an adoptive discretization is not needed.
To integrate travel times Simpson's rule is used, a fixed samples quadrature scheme.
Therefore, it is necessary to discretize the great circle segment for all station pairs.
All segments are sampled at almost the same spacing.
The increment is chosen as
\begin{equation}
    \Delta\sphericalangle = \frac 1 \SFWminsamples \min\left\{ \sphericalangle_u^v \mid u,v \in S \right\}
    \approx \SFWdeltaangle \,{}^\circ
\end{equation}
i.e.~a portion of the shortest inter-station distance.

For plotting purposes only, there is a regular $\SFWngrid \times \SFWngrid$ grid covering the domain of interest.

\worries{My implementation comes with quite a number of duplicates.
In total I am considering $\SFWnpts$ grid points.
It were enough to consider station locations just once saving 360 points.}


\section{Correlation Kernel}

Typically, the Gaussian kernel is expressed in terms of the euclidean distance.
As we consider waves traveling along the surface only, the great circle segment is the desired measure of distance.
At the surface, the shortest distance between two points $u$, $v$ is given by
\begin{equation}
    d(u,v) = \sphericalangle_u^v r_E
\end{equation}
with values in $[0, \pi R]$.
The so called great circle distance $d$ shall serve as metric and the correlation kernel of choice reads
\begin{equation}
    K(u,v) = \tau^2 \exp\!\left(-\frac 12 \frac{d(u,v)^2}{\ell^2}\right)
\end{equation}
with characteristic length $\ell$ and variance $\tau^2$.
%That Kernel has an upper and lower bound
%\begin{equation}
%    1 \geq K(x_i,x_j) \geq \tau^2 e^{-\frac12\frac{R^2\pi^2}{\ell^2}} > 0
%\end{equation}
%To proof $K$ is PD consider arbitrary wights $\alpha_i \in \mathbb R$ and locations $x_i \in \mathbb S_{r_E}$ and we have
%\begin{equation}
%    \sum_{ij} \alpha_i K(x_i, x_j) \alpha_j >
%    \varepsilon \sum_{ij} \alpha_i \alpha_j =
%    \varepsilon \sum_i 1 \alpha_i \sum_j 1 \alpha_j =
%    \langle 1,\alpha \rangle^2 > 0
%\end{equation}
%where $\varepsilon$ refers to the lower bound of $K$.

\worries{Matthias suggested to use a Poisson type kernel.
\begin{equation}
    u(x) \propto \iint \frac{R-|x|}{|x-\xi|^3} u(\xi) \, \mathrm d \xi
\end{equation}
Consider white noise at some reference radius $R$ and the covariance is
\begin{equation}
    K(x,y) \propto \iint \frac{R-|x|}{|x-\xi|^3} \frac{R-|y|}{|y-\xi|^3} \, \mathrm d \xi
\end{equation}
the choice of $R$ determines the regularity. }


\begin{figure}
    \input{def_correlation}
    \centering
    \input{fig_correlation.pgf}
    \caption{Correlation amongst a constant velocity model and a travel time observation.
        The kernel's characteristic length is $\ell=\SFWcorrell\,m$ and the path discretization is $\Delta = \SFWcorrds\,{}^\circ$. }
    \label{fig:correlation}
\end{figure}


\section{A\,Priori Model}

Let assume an a\,priori model of constant mean velocity.
Then, four parameters need to be determined before any inversion may be carried out.
That are the error level $\varepsilon$ and the kernels characteristic length $\tau$, variance $\sigma$ and mean velocity $\mu_C$.
\worries{For now I pass on that \dots }

The values I use are
\begin{align}
    \varepsilon &= \SFWepsilon \, s \;,&
    \tau &= \SFWtau \, \sfrac ms \;,&
    \ell &= \SFWell \, m \;,&
    \mu_C &= \SFWmuCpri \, \sfrac ms \;.
\end{align}


\section{Example}

it were interesting to see how the misfit successively evolves \dots
\begin{figure}
    \centering
    \input{fig_example.pgf}
    \caption{Just considering halve the stations due to heavy memory consumption }
    \label{fig:example}
\end{figure}

\begin{figure}
    \centering
    \input{fig_example_var.pgf}
    \caption{posterior standard deviation; artifacts are due to single precision }
    \label{fig:example_var}
\end{figure}

\end{document}

