\documentclass[11pt]{article}

\usepackage[utf8]{inputenc}
\usepackage[a4paper, margin=2cm]{geometry}
\usepackage{amssymb}
\usepackage{mathtools}
\usepackage{hyperref}
\usepackage{pgfplots}
\usepackage{color}

\newcommand{\mypath}{.}
\newcommand\worries[1]{\textcolor{red}{#1}}

\title{Correlation Based Surface Wave Tomography}
\author{Stefan Mauerberger}

\begin{document}
\maketitle
\begin{abstract}
    Just a summary of formulas I use \dots
\end{abstract}

\section{Setting}

The idea in my mind is to use ambient noise records for surface wave tomography.
Average slownesses amongst stations are obtained from cross correlations.
Only a narrow frequency band shall be considered to work around dispersion.
Diffraction is neglected to simplify things even further.
Then, the ray path between stations is approximated by a great circle segment.
Travel times amongst stations are given by distances divided by average slowness.

To proceed with a partly realistic setting the station geometry is borrowed from the NORSA Array.
A constant velocity perturbed by two blobs is used as a reference.
Figure~\ref{fig:path_coverage} depicts the reference velocity model with the path coverage atop.

\begin{figure}
    \centering
    \input{fig_path_coverage.pgf}
    \caption{Path coverage of the NORSA Array}
    \label{fig:path_coverage}
\end{figure}

Synthetic travel times are obtained by integrating the slowness along the ray path.
A set of pseudo observations is generated by corrupting synthetic travel times by normal noise.


\section{Line Integral along a Great Circle Segment}
In the oversimplified setting the ray path is given by the shortest distance between two points $u$, $v$ on the surface of a sphere.
The central angle is given by
\begin{align}
    \cos \sphericalangle_u^v &= \frac{u \cdot v}{|u||v|} &
    &\text{or}&
    \sin \sphericalangle_u^v &= \frac{|u \times v|}{|u||v|}
    \; .
\end{align}
To integrate along the ray path, a parametrizing equation of the grate circle passing through $u$ and $v$ is required.
With respect to a Cartesian coordinate frame such a parametrization is of the form
\begin{equation}
    r(t) = u \cos t + w \sin t
\end{equation}
with $t \in [0:2\pi$] and to be determined $w$.
For $w$ we find
\begin{align}
    v &= r(t = \sphericalangle_u^v) &
    &\leadsto &
    w &= \frac{v - u \cos \sphericalangle_u^v}{\sin \sphericalangle_u^v}
    \; .
\end{align}
The segment between $u$ and $v$ is given by $C= \{r(t) \mid t \in [0, \sphericalangle_u^v]\}$.
Travel times are calculated by carrying out the line integral along the great circle segment
\begin{equation}
    T_{u,v}[s]
    = \int_C \frac1{c(r)} \mathrm d r
    = \int_0^{\sphericalangle_u^v} \frac 1{c(r(t))} |\acute r(t)| \mathrm d t
\end{equation}
where $c$ refers to a velocity model.
The length of a line element on a great circle is given by
\begin{equation}
    |\acute c(t)| = |w \cos t - u \sin t| = ??? = r_E
    \; .
\end{equation}

\section{Discretization}

To integrate travel times a fixed samples quadrature scheme shall be used.
It is necessary to discretize the great circle segment of all pairs of stations.
An equal spacing is used and the approximate increment is a third of the smallest central angle.
Simpson's rule is used for numeric integration.
\worries{That discretization comes with quite a number of duplicates.
It were enough to consider station locations just once.}


\section{Correlation Kernel}

Typically, the Gaussian kernel is expressed in terms of the euclidean distance.
As we assume surface waves the great circle segment is the desired measure of distance.
At the surface, the shortest distance between two points $u$, $v$ is given by
\begin{equation}
    d(u,v) = \sphericalangle_u^v r_E
\end{equation}
with values in $[0, \pi R]$.
The so called great circle distance $d$ shall serve as metric and the correlation kernel of choice reads
\begin{equation}
    K(u,v) = \tau^2 \exp\!\left(-\frac 12 \frac{d(u,v)^2}{\ell^2}\right)
\end{equation}
with characteristic length $\ell$ and variance $\tau^2$.
That Kernel has an upper and lower bound
\begin{equation}
    1 \geq K(x_i,x_j) \geq \tau^2 e^{-\frac12\frac{R^2\pi^2}{\ell^2}} > 0
\end{equation}
To proof $K$ is PD consider arbitrary wights $\alpha_i \in \mathbb R$ and locations $x_i \in \mathbb S_{r_E}$ and we have
\begin{equation}
    \sum_{ij} \alpha_i K(x_i, x_j) \alpha_j >
    \varepsilon \sum_{ij} \alpha_i \alpha_j =
    \varepsilon \sum_i 1 \alpha_i \sum_j 1 \alpha_j =
    \langle 1,\alpha \rangle^2 > 0
\end{equation}
where $\varepsilon$ refers to the lower bound of $K$.


\section{A priori Model}
\dots

\begin{figure}
    \centering
    \input{fig_correlation.pgf}
    \caption{Correlation amongst velocity and station  }
    \label{fig:correlation}
\end{figure}


\end{document}

