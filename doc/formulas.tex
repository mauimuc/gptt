\documentclass{article}

\usepackage[T1]{fontenc}
\usepackage[a4paper]{geometry}

\usepackage{amssymb}
\usepackage{mathtools}
\usepackage{hyperref}

\newcommand{\mypath}{.}

\title{Formulas}
\author{Stefan Mauerberger}

\begin{document}
\maketitle
\begin{abstract}
    Just a summary of formulas I use \dots
\end{abstract}

\section{Great Circle Distance}
The shortest distance between two points $s$, $t$ on the surface of a sphere is given by
\begin{equation}
    d = r \sphericalangle_s^t
\end{equation}
where $\sphericalangle$ is termed central angle and $r$ is the radius.
The central angle is given by
\begin{equation}
    \cos \sphericalangle_s^t = \frac{s \cdot t}{|s||t|}
\end{equation}
and the inner product in spherical coordinates reads
\begin{equation}
    s \cdot t =
|s||t| \left( \sin \theta_s \sin \theta_t \cos (\phi_s - \phi_t) + \cos \theta_s \cos \theta_t \right )
    \; .
\end{equation}
where $\varphi$ refers to longitude and $\theta$ to co-latitude.
References: \url{https://en.wikipedia.org/wiki/Great-circle_distance} and \url{https://en.wikipedia.org/wiki/Spherical_law_of_cosines}

\section{Great Circle Path}
Parametric expression along a great circle
Consider two points $s$ and $t$ located at the Earth's surface.
Required is an equation parametrizing the grate circle passing through $s$ and $r$.


\section{Gaussian Kernel}
The great circle distance shall serve as metric.
Te Gaussian kernel is given by
\begin{equation}
    K(OC\mathbf {x} ,\mathbf {x'})   =
    \exp \left(-{\frac {\|\mathbf {x} -\mathbf {x'} \|^{2}}{2\sigma ^{2}}}\right)
\end{equation}


\end{document}

