\documentclass{article}

\usepackage[T1]{fontenc}
\usepackage[a4paper]{geometry}

\usepackage{amssymb}
\usepackage{mathtools}
\usepackage{hyperref}

\newcommand{\mypath}{.}

\title{Formulas}
\author{Stefan Mauerberger}

\begin{document}
\maketitle
\begin{abstract}
    Just a summary of formulas I use \dots
\end{abstract}


\section{Line Integral along a Great Circle Segment}
To keep things simple let neglect diffraction and consider surface waves only.
Then, the ray path is given by the shortest distance between two points $u$, $v$ on the surface of a sphere.
The central angle is given by
\begin{align}
    \cos \sphericalangle_u^v &= \frac{u \cdot v}{|u||v|} &
    &\text{or}&
    \sin \sphericalangle_u^v &= \frac{|u \times v|}{|u||v|}
    \; .
\end{align}
To integrate a travel times, a parametrizing equation of the grate circle passing through $u$ and $v$ is required.
With respect to a Cartesian coordinate frame such a parametrization is of the form
\begin{equation}
    c(t) = u \cos t + w \sin t
\end{equation}
with $t \in [0:2\pi$] and to be determined $w$.
To determine $w$ we solve
\begin{equation}
    v = c(t = \sphericalangle_u^v)
\end{equation}
and we find
\begin{equation}
    w = \frac{v - u \cos \sphericalangle_u^v}{\sin \sphericalangle_u^v}
    \; .
\end{equation}
The segment between $u$ and $v$ is given by $\{c(t) \mid t \in [0, \sphericalangle_u^v]\}$.
The line integral along the great circle segment $\sphericalangle_u^v$ is required to calculate the travel time.
With the above parametrization we find
\begin{equation}
    T_{u,v}[s]
    = \int_0^{\sphericalangle_u^v} \frac 1{s(c(t))} |\acute c(t)| \mathrm d t
\end{equation}
and the length of a line element reads
\begin{equation}
    |\acute c(t)| = |w \cos t - u \sin t|
\end{equation}
\dots



\section{Gaussian Kernel}
The great circle distance shall serve as metric.
Te Gaussian kernel is given by
\begin{equation}
    K(\mathbf {x} ,\mathbf {x'})   =
    \exp \left(-{\frac {\|\mathbf {x} -\mathbf {x'} \|^{2}}{2\sigma ^{2}}}\right)
\end{equation}


\end{document}

